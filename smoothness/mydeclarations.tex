% http://stackoverflow.com/questions/3208691/how-to-identify-equation-and-subsection-counters-in-latex
% \newtheorem{myDefinition}[section]{Definition}
\newtheorem{myDefinition}{Definition}
\newtheorem{myTheorem}{Theorem}
\newtheorem{myLemma}{Lemma}

\numberwithin{equation}{section}
\numberwithin{myTheorem}{section}

% other definitions
\input{\fullpath/colors}
\newcommand{\paren}[1]   { \left(  #1 \right) }
\newcommand{\inner}[2]   { \langle #1 \overline{#2} \rangle }
\newcommand{\innerx}[2]  { \langle #1 | #2 \rangle }
\newcommand{\brac}[1]    { \left[  #1 \right] }
\newcommand{\lst}[1]     { \left\{ #1 \right\} }

\endinput  %  -  -  - -  -  - -  -  - -  -  - -  -  - -  -  - -  -  - -  -  -
\input{\fullpath/"matrix basics"}
\newcommand{\abs}[1]     {\left| #1 \right|}
\newcommand{\norm}[1]    {\left\lVert #1 \right\rVert}
\newcommand{\normo}[1]   {\left\lVert #1 \right\rVert_{1}}
\newcommand{\normt}[1]   {\left\lVert #1 \right\rVert_{2}}
\newcommand{\normi}[1]   {\left\lVert #1 \right\rVert_\infty}
\newcommand{\normp}[1]   {\left\lVert #1 \right\rVert_{p}}
\newcommand{\normf}[1]   {\left\lVert #1 \right\rVert_F}
\newcommand{\normtL}[1]  {\left\lVert #1 \right\rVert_{L^{2}}}
\newcommand{\normtl}[1]  {\left\lVert #1 \right\rVert_{l^{2}}}

\newcommand{\norms}[1]   {\left\lVert #1 \right\rVert^{2}}
\newcommand{\normos}[1]  {\left\lVert #1 \right\rVert_{1}^{2}}
\newcommand{\normts}[1]  {\left\lVert #1 \right\rVert_{2}^{2}}
\newcommand{\normis}[1]  {\left\lVert #1 \right\rVert_\infty^{2}}
\newcommand{\normps}[1]  {\left\lVert #1 \right\rVert_{p}^{2}}
\newcommand{\normfs}[1]  {\left\lVert #1 \right\rVert_F^{2}}
\newcommand{\normtLs}[1] {\left\lVert #1 \right\rVert_{L^{2}}^{2}}
\newcommand{\normtls}[1] {\left\lVert #1 \right\rVert_{l^{2}}^{2}}


\endinput %-------------------------------
\input{\fullpath/spaces}

% matrices  --   --   --   --   --   --   --   --   --   --   --   --
\newcommand{\A}[1]       { \mathbf{A}^{\rm{#1}} }
\newcommand{\As}[0]      { \mathbf{A}^{\rm{*}} }
\newcommand{\PP}[0]      { \mathbf{P} }
\newcommand{\R}[0]       { \mathbf{R} }
\newcommand{\V}[0]       { \mathbf{V} }
\newcommand{\I}[1]       { \mathbf{I}_{#1} }

\newcommand{\re}[0]      { \text{Re} }
\newcommand{\im}[0]      { \text{Im} }

\newcommand{\half}[0]    { \frac{1} {2} }

% domains  --   --   --   --   --   --   --   --   --   --   --   --
\newcommand{\bigL}[0]    { L^{2} }
\newcommand{\littlel}[0] { l^{2} }

\newcommand{\innerxL}[2] { \langle #1 | #2 \rangle_{\bigL} }
\newcommand{\innerxl}[2] { \langle #1 | #2 \rangle_{\littlel} }

% action over domains
\newcommand{\domL}[0]    { \Omega }
\newcommand{\doml}[0]    { \sigma }

\newcommand{\intcont}[0] { \Omega = \lst{x\in\real{}\colon -1 \le x \le 1} }
\newcommand{\intdisc}[0] { \sigma = \lst{x\colon -1 \le x \le 1} }

\newcommand{\spaceL}[0]  { \bigL (\Omega) }
\newcommand{\spacel}[0]  { \littlel \paren{\sigma} }

\newcommand{\intL}[0]    { \bigL[-1,1]}
\newcommand{\intl}[0]    { \littlel[-1,1] }

% \newcommand{\intdom}[1]  { \int_{\domL} #1(x) \overline{#1}^{*}(x) dx }
% \newcommand{\sumdom}[1]  { \sum_{x\in \doml} #1(x) #1^{*}(x) }
\newcommand{\intdom}[1]  { \int_{\domL}      #1(x) \overline{#1(x)} dx }
\newcommand{\sumdom}[1]  { \sum_{x\in \doml} #1(x) \overline{#1(x)} }

\newcommand{\intdomx}[2] { \int_{\domL} #1(x) \overline{#2(x)} dx }
\newcommand{\sumdomx}[2] { \sum_{x\in \doml}  \overline{#2(x)} \Delta }

% epsilon 
\newcommand{\eps}[0]     {\epsilon}
\newcommand{\egtz}[0]    {\eps > 0}

% stray commands  --   --   --   --   --   --   --   --   --   --   --   --

\newcommand{\sgn}[0]     { \text{sgn } }

% bound angle
\newcommand{\dtwo}[0]    { \overline{\dtwoopen} }
\newcommand{\dtwoopen}[0]{ D_{2} }

% order
\newcommand{\order}[1]   { \mathcal{o}\paren{#1} }
\newcommand{\Order}[1]   { \mathcal{O}\paren{#1} }

\newcommand{\mx}[2]      { \max\limits_{#1 \in \cmplx{ #2 }} }
\newcommand{\mn}[2]      { \min\limits_{#1 \in \cmplx{ #2 }} }

\newcommand{\mxxn}[0]    { \mx{x}{n} }
\newcommand{\mnxn}[0]    { \mn{x}{n} }

\newcommand{\mxball}[0]  { \max\limits_{\normt{x} = 1} }

% is equal?
\newcommand{\iseq}[0]    { \overset{?}{=} }

% map A
\newcommand{\mapa}[1]    { \overset{\A{#1}}{\mapsto} }

% header
\newcommand{\head}[1]    { $\A{}$ & $=$ & $\U{}$ & $\sig{}$ & $\V{*}$ }

% spacings and such
\newcommand{\wfour}[0]   {1.05in}

\endinput %-------------------------------