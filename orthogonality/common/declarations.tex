%\input{\fullpath delimiters}
%\input{\fullpath summations}

% other definitions
\input{\pathname/art}
\input{\fullpath/colors}
\input{\fullpath/code}
\newcommand{\paren}[1]   { \left(  #1 \right) }
\newcommand{\inner}[2]   { \langle #1 \overline{#2} \rangle }
\newcommand{\innerx}[2]  { \langle #1 | #2 \rangle }
\newcommand{\brac}[1]    { \left[  #1 \right] }
\newcommand{\lst}[1]     { \left\{ #1 \right\} }

\endinput  %  -  -  - -  -  - -  -  - -  -  - -  -  - -  -  - -  -  - -  -  -
\input{\pathname/"matrix basics"}
\newcommand{\abs}[1]     {\left| #1 \right|}
\newcommand{\norm}[1]    {\left\lVert #1 \right\rVert}
\newcommand{\normo}[1]   {\left\lVert #1 \right\rVert_{1}}
\newcommand{\normt}[1]   {\left\lVert #1 \right\rVert_{2}}
\newcommand{\normi}[1]   {\left\lVert #1 \right\rVert_\infty}
\newcommand{\normp}[1]   {\left\lVert #1 \right\rVert_{p}}
\newcommand{\normf}[1]   {\left\lVert #1 \right\rVert_F}
\newcommand{\normtL}[1]  {\left\lVert #1 \right\rVert_{L^{2}}}
\newcommand{\normtl}[1]  {\left\lVert #1 \right\rVert_{l^{2}}}

\newcommand{\norms}[1]   {\left\lVert #1 \right\rVert^{2}}
\newcommand{\normos}[1]  {\left\lVert #1 \right\rVert_{1}^{2}}
\newcommand{\normts}[1]  {\left\lVert #1 \right\rVert_{2}^{2}}
\newcommand{\normis}[1]  {\left\lVert #1 \right\rVert_\infty^{2}}
\newcommand{\normps}[1]  {\left\lVert #1 \right\rVert_{p}^{2}}
\newcommand{\normfs}[1]  {\left\lVert #1 \right\rVert_F^{2}}
\newcommand{\normtLs}[1] {\left\lVert #1 \right\rVert_{L^{2}}^{2}}
\newcommand{\normtls}[1] {\left\lVert #1 \right\rVert_{l^{2}}^{2}}


\endinput %-------------------------------
\input{\pathname/spaces}

% stray commands
\newcommand{\authorTopa}[0]  {\href{mailto:Daniel.M.Topa@erdc.dren.mil}{Daniel Topa}}
\newcommand{\authorEmbid}[0] {\href{mailto:pfembid@math.unm.edu}{Pedro Embid}}
\newcommand{\engility}[0]    {Engility Corp.\\USACE Engineer Research and Development Center\\Vicksburg MS}

% http://tex.stackexchange.com/questions/34921/how-to-overlap-images-in-a-beamer-slide
\def\Put(#1,#2)#3{\leavevmode\makebox(0,0){\put(#1,#2){#3}}}


\newenvironment{myfancyblock}%
{\begin{center}\begin{tikzpicture}}%
{\end{tikzpicture}\end{center}}%

\newcommand{\opaqueblock}[4]{
\node<#1>[#2=#3] (X) {#4};
}

\newcommand{\onedot}[0]  
{  
  \begin{textblock*}{1cm}(1.136\textwidth,8.9cm)%
   {\color{medgray}{.}}
  \end{textblock*} 
}

\newcommand{\twodots}[0]  
{  
  \begin{textblock*}{1cm}(1.136\textwidth,8.9cm)%
   {\color{medgray}{..}}
  \end{textblock*} 
}

\newcommand{\donedot}[0]  
{  
  \begin{textblock*}{1cm}(1.136\textwidth,8.9cm)%
   {\color{darkgray}{.}}
  \end{textblock*} 
}

\newcommand{\dtwodots}[0]  
{  
  \begin{textblock*}{1cm}(1.136\textwidth,8.9cm)%
   {\color{darkgray}{..}}
  \end{textblock*} 
}

% domains  --   --   --   --   --   --   --   --   --   --   --   --
\newcommand{\bigL}[0]    { L^{2} }
\newcommand{\littlel}[0] { l^{2} }

\newcommand{\innerxL}[2] { \langle #1 | #2 \rangle_{\bigL} }
\newcommand{\innerxl}[2] { \langle #1 | #2 \rangle_{\littlel} }

% action over domains
\newcommand{\domL}[0]    { \Omega }
\newcommand{\doml}[0]    { \sigma }

\newcommand{\intcont}[0] { \Omega = \lst{x\in\real{}\colon -1 \le x \le 1} }
\newcommand{\intdisc}[0] { \sigma = \lst{x\colon -1 \le x \le 1} }

\newcommand{\spaceL}[0]  { \bigL (\Omega) }
\newcommand{\spacel}[0]  { \littlel \paren{\sigma} }

\newcommand{\intL}[0]    { \bigL[-1,1]}
\newcommand{\intl}[0]    { \littlel[-1,1] }

\newcommand{\intdom}[1]  { \int_{\domL} #1(x) #1^{*}(x) dx }
\newcommand{\sumdom}[1]  { \sum_{x\in \doml} #1(x) #1^{*}(x) \Delta }

\newcommand{\intdomx}[2] { \int_{\domL} #1(x) #2^{*}(x) dx }
\newcommand{\intdomy}[2] { \bl{\int_{\domL}} #1(x) #2^{*}(x) dx }

\newcommand{\sumdomx}[2] { \sum_{x\in \doml} #1(x) #2^{*}(x) \Delta }
\newcommand{\sumdomy}[2] { \sum_{x\in \doml} #1(x) #2^{*}(x) \Delta }
\newcommand{\sumdomz}[2] { \bl{\sum_{x\in \doml}} #1(x) #2^{*}(x)\Delta }

%% Title page
\newcommand{\tpage}[0] 
{
\begin{frame}
  \begin{center}
    \bl{{\LARGE{Orthogonality and Computation}}} \\[10pt]
    {\href{mailto:daniel.m.topa@erdc.dren.mil}{\large{Daniel Topa$^{1,2,3}$ and \href{http://math.unm.edu/people/academic-personnel/pedro-embid}{Pedro Embid}}$^{1}$}} \\[10pt]
    %
    \mg{{\footnotesize{$^{1}$ \href{http://www.unm.edu}{University of New Mexico}, \\\href{http://math.unm.edu}{Department of Mathematics and Statistics}}}} \\
    \mg{{\footnotesize{$^{2}$ \href{https://www.lanl.gov}{Los Alamos National Laboratory}}}} \\[3pt]
    \mg{{\footnotesize{$^{3}$ \href{http://www.engilitycorp.com}{Engility Corporation}, \\\href{http://www.usace.army.mil}{USACE} \href{http://www.erdc.usace.army.mil}{Engineer Research and Development Center},\\\href{http://www.erdc.usace.army.mil/Locations/InformationTechnologyLaboratory.aspx}{Information Technology Laboratory}}}} \\[10pt]
    %
    {\normalsize{\href{http://www.world-academy-of-science.org/worldcomp15/ws/conferences/csc15}{CSC'15\\13th International Conference on Scientific Computing}}} \\[5pt]
    {\normalsize{29 July 2015}}
    \Put(-200,200){\href{http://www.world-academy-of-science.org/worldcomp15/ws/conferences/csc15}{\includegraphics[ height = 2cm ]{../graphics/"logo csc"}}}
  \end{center}
\end{frame}
}

% https://tex.stackexchange.com/questions/6370/how-to-center-a-beamercolorbox
\newcommand{\courtney}[0]
{
\begin{frame}      %  %  %   FRAME   %  %  %
\frametitle{Overview}
  %
  \begin{center}
    \includegraphics[ width = 6.75cm ]{../graphics/caution45.png}
  \end{center}
  %
  \begin{textblock*}{5cm}(3.90cm,5.65cm)%
    \begin{center}
     \LARGE{\bf{{\color{\clr}{CAUTION\\[-2pt]}}}}
     \Large{\bf{{\color{\clr}{when using\\[-3pt]orthogonal functions\\[-3pt]in discrete spaces}}}}
    \end{center}
  \end{textblock*}
  %
  %
\end{frame}
}

\newcommand{\courtneyX}[0]
{
\begin{frame}      %  %  %   FRAME   %  %  %
\frametitle{Overview}
  %
  \begin{center}
    {\transparent{0.2}\includegraphics[ width = 6.75cm ]{../graphics/caution.png}}
  \end{center}
  %
  \begin{textblock*}{5cm}(3.90cm,5.65cm)%
    \begin{center}
     \LARGE{\bf{{\color{\clr}{CAUTION\\[-2pt]}}}}
     \Large{\bf{{\color{\clr}{when using\\[-3pt]orthogonal functions\\[-3pt]in discrete spaces}}}}
    \end{center}
  \end{textblock*}
  %
  %
\end{frame}
}


\newcommand{\final}[0]
{
{\setbeamercolor{background canvas}{bg=black}
\color{darkgray}
\begin{frame}      %  %  %   FRAME   %  %  %
\gr{Linearly independent system}
\frametitle{Off-diagonal entries persist in $l^{2}$}
  %
\begin{equation*}
  %\begin{split}
    \mat{cccc}{
    \dg{\innerx{g_{0}}{g_{0}}} & \gr{\innerx{g_{0}}{g_{1}}} & \cdots & \gr{\innerx{g_{0}}{g_{d}}} \\
    \gr{\innerx{g_{1}}{g_{0}}} & \dg{\innerx{g_{1}}{g_{1}}} & \cdots & \gr{\innerx{g_{1}}{g_{d}}} \\
    \vdots & \vdots & \ddots & \vdots \\
    \gr{\innerx{g_{d}}{g_{0}}} & \gr{\innerx{g_{d}}{g_{1}}} & \cdots & \dg{\innerx{g_{d}}{g_{d}}}
    }
    \mat{c}{ b_{0} \\ b_{1} \\ \vdots \\ b_{d} }
    =
    \mat{c}{ \innerx{f}{g_{0}} \\ \innerx{f}{g_{1}} \\ \vdots \\ \innerx{f}{g_{d}} } \quad \gr{\angle}
    \label{eq:l:full}    
  %\end{split}
\end{equation*}

\rd{Orthogonal system}
\begin{equation*}
  %\begin{split}
    \mat{cccc}{
    \dg{\innerx{g_{0}}{g_{0}}} & \rzero & \cdots & \rzero \\
    \rzero & \dg{\innerx{g_{1}}{g_{1}}} & \cdots & \rzero \\
    \vdots & \vdots & \ddots & \vdots \\
    \rzero & \rzero & \cdots & \dg{\innerx{g_{d}}{g_{d}}}
    }
    \mat{c}{ b_{0} \\ b_{1} \\ \vdots \\ b_{d} }
    =
    \mat{c}{ \innerx{f}{g_{0}} \\ \innerx{f}{g_{1}} \\ \vdots \\ \innerx{f}{g_{d}} } \quad \rd{\perp}
    \label{eq:l:full}    
  %\end{split}
\end{equation*}
  %
  \dtwodots
\end{frame}
}
}

\endinput  %  -  -  -  -  -  -  -  -  -  -  -  -  -  -  -  -  -  -  -  -

%\tiny
%\scriptsize
%\footnotesize
%\small
%\normalsize
%\large
%\Large
%\LARGE
%\huge
%\Huge