\section{Spaces}

Our troubles begin when we move, implicitly, from the continuous space $\bigL$ to the discrete space of $\bigL$. That is when we move from the continuum, the theoretical realm of the chalkboard, to discrete space, the realm of computer calculation. A typical case involves the Legendre polynomials which are orthogonal over the interval $\brac{-1,1}$. In the continuum. But the instant we discretize the space as in measurement or computation, we switch to a discrete topology and sacrifice orthogonality.

To set the stage, we need to define the domains, norms and membership for both topologies. Take an arbitrary interval on the real line, $[a,b]$ where $a<b$. In the continuum we can call this domain $\Omega$ and formally define
  % % % EQUATION
  \begin{equation}
    \Omega = \lst{x\ir\colon a \le x \le b}.
    \label{eq:domain:L}
  \end{equation}
  % % %
For the discrete case we only have $\doml\in\real{n}$, an denumerable set of $n$ points
  % % % EQUATION
  \begin{equation}
    \doml = \lst{x_{1}, x_{2}, \dots, x_{n}} = \lst{x_{k}}_{k=1}^{n} .
    \label{eq:domain:l}
  \end{equation}
  % % %
These points are ordered
  % % % EQUATION
  \begin{equation}
    a \le x_{1} \le x_{2} \le \cdots \le x_{n} \le b .
  \end{equation}
  % % %
To measure distance we introduce the natural norms. For continuous topologies $x\in\Omega$ and
  % % % EQUATION
  \begin{equation}
    \norm{F(x)}_{\bigL}^{2} = \intdom{F} .
    \label{eq:norm:L}
  \end{equation}
  % % %
Integration is in the Lebesgue sense. For discrete topologies $x\in\sigma$ and
  % % % EQUATION
  \begin{equation}
    \norm{f(x)}_{\littlel}^{2} = \sumdom{f} .
    \label{eq:norm:l}
  \end{equation}
  % % %
These points $\doml$ may form a uniform mesh where $x_{k+1} - x_{k} = \Delta$, $k=1,2,\dots,n$ but this is not necessary.
For emphasis we will reserve capital letters for examples in the continuum and lower case letters for examples in the discrete spaces.

The collection of all functions which are square integrable is $\spaceL$. A function $F(x)\colon \real{}  \mapsto \cmplx{}$ is an element in this space iff the norm is finite:
  % % % EQUATION
  \begin{equation}
    F(x) \in\spaceL \quad \iff \quad \intdom{F} < \infty.
    \label{eq:member:L}
  \end{equation}
  % % %
The collection of all functions which are square summable is $\spacel$. A function $f(x)\colon \real{}  \mapsto \cmplx{}$ is an element in this space iff the norm is finite:
  % % % EQUATION
  \begin{equation}
    f(x) \in\spacel \quad \iff \quad \sumdom{f} < \infty.
    \label{eq:member:l}
  \end{equation}
  % % %

With the stage set, we turn to the task at hand: resolving a target function in the bases of a complete and linearly independent set of functions over the domain. Let the highest order in the expansion be $d$ for degree of fit. For the continuum case where $x\in\Omega$
  % % % EQUATION
  \begin{equation}
    F(x) \approx a_{0}G_{0}(x) + a_{1}G_{1}(x) + \cdots + a_{d}G_{d}(x) .
    \label{eq:approx F}
  \end{equation}
  % % %
In discrete space $x\in\sigma$ we have
  % % % EQUATION
  \begin{equation}
    f(x) \approx b_{0}g_{0}(x) + b_{1}g_{1}(x) + \cdots + b_{d}g_{d}(x) .
    \label{eq:approx G}
  \end{equation}
  % % %

\subsection{\label{subsec:continuum}Continuum}
The choice of a polynomial basis is driven by the domain.

The Riesz-Fischer theorem\cite[p. 330]{Rudin}
\newtheorem{theorem}{Theorem}
\begin{theorem}[Riesz-Fischer]
Let $\lst{\phi_{n}}$ be orthonormal sequence of functions on $\Omega$ and suppose $\sum \abs{c_{n}}$ converges. Denote the partial sum as
  % % % EQUATION
  \begin{equation*}
    s_{n} = c_{1}\phi_{1} + \cdots + c_{n}\phi_{n} .
  \end{equation*}
  % % %
There there exists a function $F\in\spaceL$ such that $\lst{s_{n}}$ converges to $F$ in $\spaceL$, and such that
  % % % EQUATION
  \begin{equation}
    F = \sum_{n=1}^{\infty} c_{n}\phi_{n}
  \end{equation}
  % % %
almost everywhere.
\end{theorem}

\endinput %  -  -  -  -  -  -  -  -  -  -  -  -  -  -  -  -  -  -