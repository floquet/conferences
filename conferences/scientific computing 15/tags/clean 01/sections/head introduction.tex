\section{Introduction}
Orthogonality and projection are two facets of the same gem and are foundation concepts in many areas of science and engineering. For example, the  mathematics of quantum mechanics and quantum field theory are the embodiment of the power of orthogonal projection.

Perhaps part of the confusion may stem from the famous case where the two spaces are connected. 
The relation between smoothness in physical space and decay of the Fourier amplitudes is so fundamental in mathematics that it gave rise to the so-called Sobolev spaces, where the smoothness of the function $f(x)$ is understood in terms of the $\bigL-$norm and corresponding decay of the coefficients is given in the related $\bigL-$norm via the Parseval identity
  % % % EQUATION
  \begin{equation}
    \int_{-\pi}^{\pi} \abs{\frac{d^{k}f}{dx^{k}}}^{2} dx = \sum_{n=-\infty}^{n=\infty} n^{2k} \abs{c_{n}}^{2} \ .
  \end{equation}
  % % %

\endinput %  -  -  -  -  -  -  -  -  -  -  -  -  -  -  -  -  -  -