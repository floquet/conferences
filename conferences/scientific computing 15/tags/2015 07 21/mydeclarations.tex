% http://stackoverflow.com/questions/3208691/how-to-identify-equation-and-subsection-counters-in-latex
% \newtheorem{myDefinition}[section]{Definition}
\newtheorem{myDefinition}{Definition}
\newtheorem{myTheorem}{Theorem}
\newtheorem{myLemma}{Lemma}

\numberwithin{equation}{section}
\numberwithin{myTheorem}{section}

% other definitions
%% shades of gray
\definecolor{darkgray}{gray}{0.25}
\definecolor{medgray} {gray}{0.75}
\definecolor{ltgray}  {gray}{0.90}
\definecolor{vltgray} {gray}{0.95}

%\definecolor{rangecolor} {blue}
%\definecolor{nullcolor}  {red}

%% basic color directives
\newcommand{ \bl }[1]        {{\color{blue}   {#1}}}
\newcommand{ \bk }[1]        {{\color{black}  {#1}}}
\newcommand{ \rd }[1]        {{\color{red}    {#1}}}
\newcommand{ \gr }[1]        {{\color{green}  {#1}}}
\newcommand{ \pr }[1]        {{\color{purple} {#1}}}
\newcommand{ \mg }[1]        {{\color{medgray}{#1}}}

%% numbers blue
\newcommand{ \bzero }[0]     { \bl{ 0 } }
\newcommand{ \bone }[0]      { \bl{ 1 } }
\newcommand{ \btwo }[0]      { \bl{ 2 } }
\newcommand{ \bthree }[0]    { \bl{ 3 } }
\newcommand{ \bfour }[0]     { \bl{ 4 } }
\newcommand{ \bfive }[0]     { \bl{ 5 } }
\newcommand{ \bsix }[0]      { \bl{ 6 } }
\newcommand{ \bseven }[0]    { \bl{ 7 } }
\newcommand{ \beight }[0]    { \bl{ 8 } }
\newcommand{ \bnine }[0]     { \bl{ 9 } }
\newcommand{ \bminus }[0]    { \bl{ - } }
\newcommand{ \bstar }[0]     { \bl{ * } }
\newcommand{ \bi }[0]        { \bl{ i } }
\newcommand{ \bmone }[0]     { \bl{-1 } }
\newcommand{ \bmi }[0]       { \bl{-i } }
\newcommand{ \bmo }[0]       { \bl{-1 } }

\newcommand{ \bdots }[0]     { \bl{ \dots } }


%% numbers red
\newcommand{ \rzero }[0]     { \rd{ 0 } }
\newcommand{ \rone }[0]      { \rd{ 1 } }
\newcommand{ \rtwo }[0]      { \rd{ 2 } }
\newcommand{ \rthree }[0]    { \rd{ 3 } }
\newcommand{ \rfour }[0]     { \rd{ 4 } }
\newcommand{ \rfive }[0]     { \rd{ 5 } }
\newcommand{ \rsix }[0]      { \rd{ 6 } }
\newcommand{ \rseven }[0]    { \rd{ 7 } }
\newcommand{ \reight }[0]    { \rd{ 8 } }
\newcommand{ \rnine }[0]     { \rd{ 9 } }
\newcommand{ \rminus }[0]    { \rd{ - } }
\newcommand{ \rstar }[0]     { \rd{ * } }
\newcommand{ \rmone }[0]     { \rd{-1 } }

\newcommand{ \rdots }[0]     { \rd{ \dots } }

%% numbers medium gray
\newcommand{ \gzero }[0]     { \mg{ 0 } }
\newcommand{ \gone }[0]      { \mg{ 1 } }
\newcommand{ \gtwo }[0]      { \mg{ 2 } }
\newcommand{ \gthree }[0]    { \mg{ 3 } }
\newcommand{ \gfour }[0]     { \mg{ 4 } }
\newcommand{ \gfive }[0]     { \mg{ 5 } }
\newcommand{ \gsix }[0]      { \mg{ 6 } }
\newcommand{ \gseven }[0]    { \mg{ 7 } }
\newcommand{ \geight }[0]    { \mg{ 8 } }
\newcommand{ \gnine }[0]     { \mg{ 9 } }
\newcommand{ \gminus }[0]    { \mg{ - } }
\newcommand{ \goplus }[0]    { \mg{ \oplus } }

%% numbers black
\newcommand{ \bs }[0]        { {\bk{*}} }
\newcommand{ \bkzero }[0]    { {\bk{0}} }

% nums
\newcommand{ \bnum }[0]      { \bl{ \num } }
\newcommand{ \rnum }[0]      { \rd{ \num } }
\newcommand{ \gnum }[0]      { \mg{ \num } }

\endinput  %  -  -  -  -  -  -  -  -  -  -  -  -  -  -  -  -  -  -  -  -
\newcommand{\paren}[1]   { \left(  #1 \right) }
\newcommand{\inner}[2]   { \langle #1 \overline{#2} \rangle }
\newcommand{\innerx}[2]  { \langle #1 | #2 \rangle }
\newcommand{\brac}[1]    { \left[  #1 \right] }
\newcommand{\lst}[1]     { \left\{ #1 \right\} }

\endinput  %  -  -  - -  -  - -  -  - -  -  - -  -  - -  -  - -  -  - -  -  -
\input{\fullpath/"matrix basics"}
\newcommand{\abs}[1]     {\left| #1 \right|}
\newcommand{\norm}[1]    {\left\lVert #1 \right\rVert}
\newcommand{\normo}[1]   {\left\lVert #1 \right\rVert_{1}}
\newcommand{\normt}[1]   {\left\lVert #1 \right\rVert_{2}}
\newcommand{\normi}[1]   {\left\lVert #1 \right\rVert_\infty}
\newcommand{\normp}[1]   {\left\lVert #1 \right\rVert_{p}}
\newcommand{\normf}[1]   {\left\lVert #1 \right\rVert_F}
\newcommand{\normtL}[1]  {\left\lVert #1 \right\rVert_{L^{2}}}
\newcommand{\normtl}[1]  {\left\lVert #1 \right\rVert_{l^{2}}}

\newcommand{\norms}[1]   {\left\lVert #1 \right\rVert^{2}}
\newcommand{\normos}[1]  {\left\lVert #1 \right\rVert_{1}^{2}}
\newcommand{\normts}[1]  {\left\lVert #1 \right\rVert_{2}^{2}}
\newcommand{\normis}[1]  {\left\lVert #1 \right\rVert_\infty^{2}}
\newcommand{\normps}[1]  {\left\lVert #1 \right\rVert_{p}^{2}}
\newcommand{\normfs}[1]  {\left\lVert #1 \right\rVert_F^{2}}
\newcommand{\normtLs}[1] {\left\lVert #1 \right\rVert_{L^{2}}^{2}}
\newcommand{\normtls}[1] {\left\lVert #1 \right\rVert_{l^{2}}^{2}}


\endinput %-------------------------------
% bys
\newcommand{\by}[2]      {#1 \times #2}
\newcommand{\byy}[1]     {#1 \times #1}
\newcommand{\bymn}[0]    {\by{m}{n}}
\newcommand{\bymm}[0]    {\byy{m}}
\newcommand{\bynn}[0]    {\byy{n}}
\newcommand{\bynm}[0]    {\by{n}{m}}
\newcommand{\bymr}[0]    {\by{m}{\rho}}
\newcommand{\byrn}[0]    {\by{\rho}{n}}

% vector spaces
\newcommand{\real}[1]    {\mathbb{R}^{#1}}
\newcommand{\cmplx}[1]   {\mathbb{C}^{#1}}
\newcommand{\either}[1]  {\cmplx{#1}}
\newcommand{\ir}[0]      {\in\real{}}
\newcommand{\ic}[0]      {\in\cmplx{}}
\newcommand{\icm}[0]     {\in\cmplxm}
\newcommand{\icn}[0]     {\in\cmplxn}
\newcommand{\icmn}[0]    {\in\cmplxmn}
\newcommand{\irmn}[0]    {\in\realmn}
\newcommand{\icmnr}[0]   {\in\cmplxmnr}
\newcommand{\ints}[0]    {\mathbb{Z}}
\newcommand{\natnum}[0]  {\mathbb{N}}

\newcommand{\iints}[0]   {\in \mathbb{Z}}
\newcommand{\inatnum}[0] {\in \mathbb{N}}

\newcommand{\realall}[3] {\real{\by{#1}{#2}}_{#3} }
\newcommand{\cmplxall}[3]{\cmplx{\by{#1}{#2}}_{#3} }

\newcommand{\realn}[0]   {\real{n}}
\newcommand{\realm}[0]   {\real{m}}
\newcommand{\realmn}[0]  {\real{\bymn}}
\newcommand{\realnn}[0]  {\real{\byy{n}}}
\newcommand{\realmm}[0]  {\real{\byy{m}}}
\newcommand{\realmmr}[0] {\real{\byy{m}}_{\rho}}
\newcommand{\realmmm}[0] {\realmm_{m}}

\newcommand{\cmplxn}[0]  {\cmplx{n}}
\newcommand{\cmplxm}[0]  {\cmplx{m}}
\newcommand{\cmplxnn}[0] {\cmplx{\byy{n}}}
\newcommand{\cmplxmm}[0] {\cmplx{\byy{m}}}
\newcommand{\cmplxmn}[0] {\cmplx{\bymn}}
\newcommand{\cmplxmr}[0] {\cmplx{\bymr}}
\newcommand{\cmplxrn}[0] {\cmplx{\byrn}}
\newcommand{\cmplxmnr}[0]{\cmplx{\bymn}_{\rho}}
\newcommand{\cmplxmmm}[0]{\cmplxmm_{m}}
\newcommand{\cmplxmnm}[0]{\cmplxmn_{m}}
\newcommand{\cmplxnnr}[0]{\cmplxnn_{\rho}}
\newcommand{\cmplxmnn}[0]{\cmplxmn_{n}}

% spans
\newcommand{\spn}[1]     {\text{sp\,} \lst{ #1 }}


\endinput %-------------------------------

% matrices  --   --   --   --   --   --   --   --   --   --   --   --
\newcommand{\A}[1]       { \mathbf{A}^{\rm{#1}} }
\newcommand{\As}[0]      { \mathbf{A}^{\rm{*}} }
\newcommand{\PP}[0]      { \mathbf{P} }
\newcommand{\R}[0]       { \mathbf{R} }
\newcommand{\V}[0]       { \mathbf{V} }
\newcommand{\I}[1]       { \mathbf{I}_{#1} }

\newcommand{\re}[0]      { \text{Re} }
\newcommand{\im}[0]      { \text{Im} }

\newcommand{\half}[0]    { \frac{1} {2} }

% domains  --   --   --   --   --   --   --   --   --   --   --   --
\newcommand{\bigL}[0]    { L^{2} }
\newcommand{\littlel}[0] { l^{2} }

\newcommand{\innerxL}[2] { \langle #1 | #2 \rangle_{\bigL} }
\newcommand{\innerxl}[2] { \langle #1 | #2 \rangle_{\littlel} }

% action over domains
\newcommand{\domL}[0]    { \Omega }
\newcommand{\doml}[0]    { \sigma }

\newcommand{\intcont}[0] { \Omega = \lst{x\in\real{}\colon -1 \le x \le 1} }
\newcommand{\intdisc}[0] { \sigma = \lst{x\colon -1 \le x \le 1} }

\newcommand{\spaceL}[0]  { \bigL (\Omega) }
\newcommand{\spacel}[0]  { \littlel \paren{\sigma} }

\newcommand{\intL}[0]    { \bigL[-1,1]}
\newcommand{\intl}[0]    { \littlel[-1,1] }

% \newcommand{\intdom}[1]  { \int_{\domL} #1(x) \overline{#1}^{*}(x) dx }
% \newcommand{\sumdom}[1]  { \sum_{x\in \doml} #1(x) #1^{*}(x) }
\newcommand{\intdom}[1]  { \int_{\domL}      #1(x) \overline{#1(x)} dx }
\newcommand{\sumdom}[1]  { \sum_{x\in \doml} #1(x) \overline{#1(x)} }

\newcommand{\intdomx}[2] { \int_{\domL} #1(x) \overline{#2(x)} dx }
\newcommand{\sumdomx}[2] { \sum_{x\in \doml}  \overline{#2(x)} \Delta }

% epsilon 
\newcommand{\eps}[0]     {\epsilon}
\newcommand{\egtz}[0]    {\eps > 0}

% stray commands  --   --   --   --   --   --   --   --   --   --   --   --

\newcommand{\sgn}[0]     { \text{sgn } }

% bound angle
\newcommand{\dtwo}[0]    { \overline{\dtwoopen} }
\newcommand{\dtwoopen}[0]{ D_{2} }

% order
\newcommand{\order}[1]   { \mathcal{o}\paren{#1} }
\newcommand{\Order}[1]   { \mathcal{O}\paren{#1} }

\newcommand{\mx}[2]      { \max\limits_{#1 \in \cmplx{ #2 }} }
\newcommand{\mn}[2]      { \min\limits_{#1 \in \cmplx{ #2 }} }

\newcommand{\mxxn}[0]    { \mx{x}{n} }
\newcommand{\mnxn}[0]    { \mn{x}{n} }

\newcommand{\mxball}[0]  { \max\limits_{\normt{x} = 1} }

% is equal?
\newcommand{\iseq}[0]    { \overset{?}{=} }

% map A
\newcommand{\mapa}[1]    { \overset{\A{#1}}{\mapsto} }

% header
\newcommand{\head}[1]    { $\A{}$ & $=$ & $\U{}$ & $\sig{}$ & $\V{*}$ }

% spacings and such
\newcommand{\wfour}[0]   {1.05in}

\endinput %-------------------------------