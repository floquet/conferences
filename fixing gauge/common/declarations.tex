%\input{\fullpath delimiters}
%\input{\fullpath summations}

% other definitions
% basic matrices
\newcommand{ \A }[1]        { \textbf{A}^{\mathrm{\!#1}} }  % pull in exponent

% inverses
\newcommand{ \Ap }[0]       { \A{\sym} }
\newcommand{ \Ainv }[0]     { \A{-1} }
\newcommand{ \Ainvs }[0]    { \A{-*} }
\newcommand{ \AinvL }[0]    { \A{-L} }
\newcommand{ \AinvR }[0]    { \A{-R} }

% projectors
\newcommand{ \ApA }[0]      { \A{\sym}\A{} }
\newcommand{ \AAp }[0]      { \A{}\,\A{\sym} }

% matrix vector products
\newcommand{ \Ax }[0]       { \A{}\,x }
\newcommand{ \Axez }[0]     { \Ax = \zero }
\newcommand{ \Axey }[0]     { \Ax = y }
\newcommand{ \Axeb }[0]     { \Ax = b }
\newcommand{ \Asyex }[0]    { \A{*}y = x }
\newcommand{ \Axmb }[0]     { \Ax - b }
\newcommand{ \Axls }[0]     { \A{}\,x_{LS} }
\newcommand{ \Avk }[0]      { \A{}\,{\bl{ v_{k} }} }

% membership
\newcommand{ \aicmm }[0]    { \A{} \in \cmplxmm }
\newcommand{ \aicmn }[0]    { \A{} \in \cmplxmn }
\newcommand{ \aicmnr }[0]   { \A{} \in \cmplxmnr }
\newcommand{ \aicmrr }[0]   { \A{} \in \cmplxmrr }
\newcommand{ \aicmmm }[0]   { \A{} \in \cmplxmmm }
\newcommand{ \aicmnm }[0]   { \A{} \in \cmplxmnm }
\newcommand{ \aicmnn }[0]   { \A{} \in \cmplxmnn }

\newcommand{ \airmn }[0]    { \A{} \in \realxmn }
\newcommand{ \airmnr }[0]   { \A{} \in \realxmnr }
\newcommand{ \airmmm }[0]   { \A{} \in \realmmm }

\newcommand{ \xicn }[0]     { x    \in \cmplxn }

% product matrices
\newcommand{ \wx }[1]       { \A{#1}  \A{} }
\newcommand{ \wy }[1]       { \A{} \, \A{#1} }

% product matrix equations
\newcommand{ \wxe }[1]      { \W{V} = \wx{#1} }
\newcommand{ \wye }[1]      { \W{U} = \wy{#1} }

\newcommand{ \wv }[0]       { \W{V} }
\newcommand{ \wu }[0]       { \W{U} }

% SVD equations
\newcommand{ \aesvd }[1]    { \A{}  = \svd{ #1 } }
\newcommand{ \aaesvd }[1]   { \A{} &= \svd{ #1 } }
\newcommand{ \aetsvd }[1]   { \A{}  = \bur{}\, \ess{}\, \bvr{ #1 } }

% SVD transpose equations
\newcommand{ \aesvdt }[1]   { \A{ #1 }  = \svdt{ #1 } }
\newcommand{ \aaesvdt }[1]  { \A{ #1 } &= \svdt{ #1 } }

% Moore-Penrose pseudoinverse equations
\newcommand{ \apempp }[1]   { \Ap   = \mpp{ #1 } }
\newcommand{ \apaempp }[1]  { \Ap  &= \mpp{ #1 } }

\newcommand{ \apemppfr }[1] { \Ap   = \mppfr{ #1 } }
\newcommand{ \apaemppfr }[1]{ \Ap  &= \mppfr{ #1 } }

% combos
\newcommand{ \ax }[0]       { \A{}\,\X{} }  % pull in 


\endinput  %  -  -  -  -  -  -  -  -  -  -  -  -  -  -  -  -  -  -  -  -
% phrases
\def \trace{\text{tr}}
\def \rank{\text{rank\,}}
\def \sgn{\text{sgn\,}}
\def \Arg{\text{Arg\,}}

\def \ns{null space}
\def \ft{Fundamental Theorem}
\def \ftola{\ft \ of Linear Algebra}
\def \asvd{singular value decomposition}
\def \type{.png}

% symbols
\def \num{\star}             % number placeholder
\def \sym{\dagger}           % general pseudoinverse symbol
\def \ssym{\paren{\dagger}}  % sigma pseudoinverse symbol

\def \ps{\phantom{-}}        % number 
\def \bang{{\rd{ \Rightarrow \Leftarrow }}} % contradiction

% abbreviations
\def \eps{\epsilon}

\endinput  %  -  -  -  -  -  -  -  -  -  -  -  -  -  -  -  -  -  -  -  -
\input{\pathname/"bold letters"}
%% shades of gray
\definecolor{darkgray}{gray}{0.25}
\definecolor{medgray} {gray}{0.75}
\definecolor{ltgray}  {gray}{0.90}
\definecolor{vltgray} {gray}{0.95}

%\definecolor{rangecolor} {blue}
%\definecolor{nullcolor}  {red}

%% basic color directives
\newcommand{ \bl }[1]        {{\color{blue}   {#1}}}
\newcommand{ \bk }[1]        {{\color{black}  {#1}}}
\newcommand{ \rd }[1]        {{\color{red}    {#1}}}
\newcommand{ \gr }[1]        {{\color{green}  {#1}}}
\newcommand{ \pr }[1]        {{\color{purple} {#1}}}
\newcommand{ \mg }[1]        {{\color{medgray}{#1}}}

%% numbers blue
\newcommand{ \bzero }[0]     { \bl{ 0 } }
\newcommand{ \bone }[0]      { \bl{ 1 } }
\newcommand{ \btwo }[0]      { \bl{ 2 } }
\newcommand{ \bthree }[0]    { \bl{ 3 } }
\newcommand{ \bfour }[0]     { \bl{ 4 } }
\newcommand{ \bfive }[0]     { \bl{ 5 } }
\newcommand{ \bsix }[0]      { \bl{ 6 } }
\newcommand{ \bseven }[0]    { \bl{ 7 } }
\newcommand{ \beight }[0]    { \bl{ 8 } }
\newcommand{ \bnine }[0]     { \bl{ 9 } }
\newcommand{ \bminus }[0]    { \bl{ - } }
\newcommand{ \bstar }[0]     { \bl{ * } }
\newcommand{ \bi }[0]        { \bl{ i } }
\newcommand{ \bmone }[0]     { \bl{-1 } }
\newcommand{ \bmi }[0]       { \bl{-i } }
\newcommand{ \bmo }[0]       { \bl{-1 } }

\newcommand{ \bdots }[0]     { \bl{ \dots } }


%% numbers red
\newcommand{ \rzero }[0]     { \rd{ 0 } }
\newcommand{ \rone }[0]      { \rd{ 1 } }
\newcommand{ \rtwo }[0]      { \rd{ 2 } }
\newcommand{ \rthree }[0]    { \rd{ 3 } }
\newcommand{ \rfour }[0]     { \rd{ 4 } }
\newcommand{ \rfive }[0]     { \rd{ 5 } }
\newcommand{ \rsix }[0]      { \rd{ 6 } }
\newcommand{ \rseven }[0]    { \rd{ 7 } }
\newcommand{ \reight }[0]    { \rd{ 8 } }
\newcommand{ \rnine }[0]     { \rd{ 9 } }
\newcommand{ \rminus }[0]    { \rd{ - } }
\newcommand{ \rstar }[0]     { \rd{ * } }
\newcommand{ \rmone }[0]     { \rd{-1 } }

\newcommand{ \rdots }[0]     { \rd{ \dots } }

%% numbers medium gray
\newcommand{ \gzero }[0]     { \mg{ 0 } }
\newcommand{ \gone }[0]      { \mg{ 1 } }
\newcommand{ \gtwo }[0]      { \mg{ 2 } }
\newcommand{ \gthree }[0]    { \mg{ 3 } }
\newcommand{ \gfour }[0]     { \mg{ 4 } }
\newcommand{ \gfive }[0]     { \mg{ 5 } }
\newcommand{ \gsix }[0]      { \mg{ 6 } }
\newcommand{ \gseven }[0]    { \mg{ 7 } }
\newcommand{ \geight }[0]    { \mg{ 8 } }
\newcommand{ \gnine }[0]     { \mg{ 9 } }
\newcommand{ \gminus }[0]    { \mg{ - } }
\newcommand{ \goplus }[0]    { \mg{ \oplus } }

%% numbers black
\newcommand{ \bs }[0]        { {\bk{*}} }
\newcommand{ \bkzero }[0]    { {\bk{0}} }

% nums
\newcommand{ \bnum }[0]      { \bl{ \num } }
\newcommand{ \rnum }[0]      { \rd{ \num } }
\newcommand{ \gnum }[0]      { \mg{ \num } }

\endinput  %  -  -  -  -  -  -  -  -  -  -  -  -  -  -  -  -  -  -  -  -
\newcommand{\paren}[1]   { \left(  #1 \right) }
\newcommand{\inner}[2]   { \langle #1 \overline{#2} \rangle }
\newcommand{\innerx}[2]  { \langle #1 | #2 \rangle }
\newcommand{\brac}[1]    { \left[  #1 \right] }
\newcommand{\lst}[1]     { \left\{ #1 \right\} }

\endinput  %  -  -  - -  -  - -  -  - -  -  - -  -  - -  -  - -  -  - -  -  -
% range and null space
\newcommand{ \atomrng }[0]   { \mathcal{R} }
\newcommand{ \atomnll }[0]   { \mathcal{N} }

\newcommand{ \rng }[1]       { \atomrng \! \paren{ #1 } }
\newcommand{ \nll }[1]       { \atomnll \! \paren{ #1 } }

\newcommand{ \rnga }[1]      { \rng{ \A{ #1 } } }
\newcommand{ \nlla }[1]      { \nll{ \A{ #1 } } }

\newcommand{ \brnga }[1]     { \bl{ \rnga{ #1 } } }
\newcommand{ \rnlla }[1]     { \rd{ \nlla{ #1 } } }
\newcommand{ \gnlla }[1]     { \mg{ \nlla{ #1 } } }

% orthogonal decomposition
\newcommand{ \decompdo }[1]  { \rnga{ #1 }  \oplus \nlla{} }
\newcommand{ \decompco }[1]  { \rnga{}      \oplus \nlla{ #1 } }

\newcommand{ \cdecompdo }[1] { \brnga{ #1 } \oplus \rnlla{} }
\newcommand{ \cdecompco }[1] { \brnga{}     \oplus \rnlla{ #1 } }

% FTOLA
\newcommand{ \ftolado }[1]   { \cmplxn =  \decompdo{ #1 } }
\newcommand{ \ftolaco }[1]   { \cmplxm =  \decompco{ #1 } }

\newcommand{ \cftolado }[1]  { \cmplxn = \cdecompdo{ #1 } }
\newcommand{ \cftolaco }[1]  { \cmplxm = \cdecompco{ #1 } }

% rank plus nullity
\newcommand{ \rpn }[0]       {\text{rank}\paren{\A{}} + \text{dim}\paren{\text{nullity}\paren{\A{}}}}

% projectors
\newcommand{ \pra }[0]       { \textbf{P}_{\brnga{}} }
\newcommand{ \pras }[0]      { \textbf{P}_{\brnga{*}} }
\newcommand{ \pnas }[0]      { \textbf{P}_{\rnlla{*}} }
\newcommand{ \pna }[0]       { \textbf{P}_{\rnlla{}} }

\endinput  %  -  -  -  -  -  -  -  -  -  -  -  -  -  -  -  -  -  -  -  -
\input{\pathname/"least squares"}
\input{\pathname/"matrix basics"}
\newcommand{\abs}[1]     {\left| #1 \right|}
\newcommand{\norm}[1]    {\left\lVert #1 \right\rVert}
\newcommand{\normo}[1]   {\left\lVert #1 \right\rVert_{1}}
\newcommand{\normt}[1]   {\left\lVert #1 \right\rVert_{2}}
\newcommand{\normi}[1]   {\left\lVert #1 \right\rVert_\infty}
\newcommand{\normp}[1]   {\left\lVert #1 \right\rVert_{p}}
\newcommand{\normf}[1]   {\left\lVert #1 \right\rVert_F}
\newcommand{\normtL}[1]  {\left\lVert #1 \right\rVert_{L^{2}}}
\newcommand{\normtl}[1]  {\left\lVert #1 \right\rVert_{l^{2}}}

\newcommand{\norms}[1]   {\left\lVert #1 \right\rVert^{2}}
\newcommand{\normos}[1]  {\left\lVert #1 \right\rVert_{1}^{2}}
\newcommand{\normts}[1]  {\left\lVert #1 \right\rVert_{2}^{2}}
\newcommand{\normis}[1]  {\left\lVert #1 \right\rVert_\infty^{2}}
\newcommand{\normps}[1]  {\left\lVert #1 \right\rVert_{p}^{2}}
\newcommand{\normfs}[1]  {\left\lVert #1 \right\rVert_F^{2}}
\newcommand{\normtLs}[1] {\left\lVert #1 \right\rVert_{L^{2}}^{2}}
\newcommand{\normtls}[1] {\left\lVert #1 \right\rVert_{l^{2}}^{2}}


\endinput %-------------------------------
\input{\pathname/"sigma matrices"}
\input{\pathname/"svd forms"}
% bys
\newcommand{\by}[2]      {#1 \times #2}
\newcommand{\byy}[1]     {#1 \times #1}
\newcommand{\bymn}[0]    {\by{m}{n}}
\newcommand{\bymm}[0]    {\byy{m}}
\newcommand{\bynn}[0]    {\byy{n}}
\newcommand{\bynm}[0]    {\by{n}{m}}
\newcommand{\bymr}[0]    {\by{m}{\rho}}
\newcommand{\byrn}[0]    {\by{\rho}{n}}

% vector spaces
\newcommand{\real}[1]    {\mathbb{R}^{#1}}
\newcommand{\cmplx}[1]   {\mathbb{C}^{#1}}
\newcommand{\either}[1]  {\cmplx{#1}}
\newcommand{\ir}[0]      {\in\real{}}
\newcommand{\ic}[0]      {\in\cmplx{}}
\newcommand{\icm}[0]     {\in\cmplxm}
\newcommand{\icn}[0]     {\in\cmplxn}
\newcommand{\icmn}[0]    {\in\cmplxmn}
\newcommand{\irmn}[0]    {\in\realmn}
\newcommand{\icmnr}[0]   {\in\cmplxmnr}
\newcommand{\ints}[0]    {\mathbb{Z}}
\newcommand{\natnum}[0]  {\mathbb{N}}

\newcommand{\iints}[0]   {\in \mathbb{Z}}
\newcommand{\inatnum}[0] {\in \mathbb{N}}

\newcommand{\realall}[3] {\real{\by{#1}{#2}}_{#3} }
\newcommand{\cmplxall}[3]{\cmplx{\by{#1}{#2}}_{#3} }

\newcommand{\realn}[0]   {\real{n}}
\newcommand{\realm}[0]   {\real{m}}
\newcommand{\realmn}[0]  {\real{\bymn}}
\newcommand{\realnn}[0]  {\real{\byy{n}}}
\newcommand{\realmm}[0]  {\real{\byy{m}}}
\newcommand{\realmmr}[0] {\real{\byy{m}}_{\rho}}
\newcommand{\realmmm}[0] {\realmm_{m}}

\newcommand{\cmplxn}[0]  {\cmplx{n}}
\newcommand{\cmplxm}[0]  {\cmplx{m}}
\newcommand{\cmplxnn}[0] {\cmplx{\byy{n}}}
\newcommand{\cmplxmm}[0] {\cmplx{\byy{m}}}
\newcommand{\cmplxmn}[0] {\cmplx{\bymn}}
\newcommand{\cmplxmr}[0] {\cmplx{\bymr}}
\newcommand{\cmplxrn}[0] {\cmplx{\byrn}}
\newcommand{\cmplxmnr}[0]{\cmplx{\bymn}_{\rho}}
\newcommand{\cmplxmmm}[0]{\cmplxmm_{m}}
\newcommand{\cmplxmnm}[0]{\cmplxmn_{m}}
\newcommand{\cmplxnnr}[0]{\cmplxnn_{\rho}}
\newcommand{\cmplxmnn}[0]{\cmplxmn_{n}}

% spans
\newcommand{\spn}[1]     {\text{sp\,} \lst{ #1 }}


\endinput %-------------------------------
% vectors
\newcommand{ \mv }[0]         {$m-$vector}
\newcommand{ \nv }[0]         {$n-$vector}
\newcommand{ \vv }[0]         {$2-$vector}
\newcommand{ \vvv }[0]        {$3-$vector}

% generic vectors
\newcommand{ \xtwo }[0]       { \mat{c}{x_{1}\\x_{2}} }
\newcommand{ \xthree }[0]     { \mat{c}{x_{1}\\x_{2}\\x_{3}} }

\newcommand{ \ytwo }[0]       { \mat{c}{y_{1}\\y_{2}} }
\newcommand{ \ythree }[0]     { \mat{c}{y_{1}\\y_{2}\\y_{3}} }

\newcommand{ \xitwo }[0]      { \mat{c}{\xi_{1}\\\xi_{2}} }
\newcommand{ \xithree }[0]    { \mat{c}{\xi_{1}\\\xi_{2}\\\xi_{3}} }

\newcommand{ \etatwo }[0]     { \mat{c}{\eta_{1}\\\eta_{2}} }
\newcommand{ \etathree }[0]   { \mat{c}{\eta_{1}\\\eta_{2}\\\eta_{3}} }

\newcommand{ \bvtwo }[0]      { \mat{c}{b_{1}\\b_{2}} }
\newcommand{ \bvthree }[0]    { \mat{c}{b_{1}\\b_{2}\\b_{3}} }

\newcommand{ \ptwo }[0]       { \mat{c}{x \\ y} }
\newcommand{ \pthree }[0]     { \mat{c}{x \\ y \\ z} }

% unit vectors
\newcommand{ \xx }[0]         { \mat{c}{1\\0} }
\newcommand{ \yy }[0]         { \mat{c}{0\\1} }

\newcommand{ \xxx }[0]        { \mat{c}{1\\0\\0} }
\newcommand{ \yyy }[0]        { \mat{c}{0\\1\\0} }
\newcommand{ \zzz }[0]        { \mat{c}{0\\0\\1} }

% rotations
\newcommand{ \vecpp }[0]      { \mat{c}{1\\1} }
\newcommand{ \vecpm }[0]      { \mat{r}{1\\-1} }
\newcommand{ \vecmp }[0]      { \mat{r}{-1\\1} }

% zeros
\newcommand{ \zero }[0]       { \textbf{0} }
\newcommand{ \zerotwo }[0]    { \mat{c}{ 0 \\ 0 } }
\newcommand{ \zerothree }[0]  { \mat{c}{ 0 \\ 0 \\ 0 } }

\newcommand{ \gzerotwo }[0]   { {\mg{ \mat{c}{ 0 \\ 0 } }} }
\newcommand{ \gzerothree }[0] { {\mg{ \mat{c}{ 0 \\ 0 \\ 0 } }} }

\newcommand{ \rzerotwo }[0]   { \rd{ \zerotwo } }
\newcommand{ \rzerothree }[0] { \rd{ \zerothree } }

\newcommand{ \rlzerotwo }[0]  { \lst{ \rd{ \zerotwo   }} }
\newcommand{ \rlzerothree }[0]{ \lst{ \rd{ \zerothree }} }

\newcommand{ \glzerotwo }[0]  { \lst{ \mg{ \zerotwo   }} }
\newcommand{ \glzerothree }[0]{ \lst{ \mg{ \zerothree }} }

% trivial
\newcommand{ \trivial }[0]    { \lst{\textbf{0}} }

% placeholders
\newcommand{ \startwo }[0]    { \mat{c}{ \num \\ \num } }
\newcommand{ \starthree }[0]  { \mat{c}{ \num \\ \num \\ \num } }

\newcommand{ \rstartwo }[0]   { \rd{ \startwo } }
\newcommand{ \rstarthree }[0] { \rd{ \starthree } }

\newcommand{ \bstartwo }[0]   { \bl{ \startwo } }
\newcommand{ \bstarthree }[0] { \bl{ \starthree } }

% unit circle
\newcommand{ \circletwo }[0]  { \mat{c}{\cos \theta \\ \sin \theta} }
\newcommand{ \circlethree }[0]{ \mat{c}{\cos \phi \, \sin \theta \\  \sin \phi \, \sin  \theta \\ \cos \theta} }
\newcommand{ \st }[0]         { S_{2}\! \paren{ \theta } }
\newcommand{ \stp }[0]        { S_{3}\! \paren{ \theta, \phi } }

\endinput

% stray commands
\newcommand{\authorTopa}[0]  {\href{mailto:Daniel.M.Topa@erdc.dren.mil}{Daniel Topa}}
\newcommand{\authorEmbid}[0] {\href{mailto:pfembid@math.unm.edu}{Pedro Embid}}
\newcommand{\engility}[0]    {Engility Corp.\\USACE Engineer Research and Development Center\\Vicksburg MS}

% http://tex.stackexchange.com/questions/34921/how-to-overlap-images-in-a-beamer-slide
\def\Put(#1,#2)#3{\leavevmode\makebox(0,0){\put(#1,#2){#3}}}


\newenvironment{myfancyblock}%
{\begin{center}\begin{tikzpicture}}%
{\end{tikzpicture}\end{center}}%

\newcommand{\opaqueblock}[4]{
\node<#1>[#2=#3] (X) {#4};
}

\newcommand{\onedot}[0]  
{  
  \begin{textblock*}{1cm}(1.136\textwidth,8.9cm)%
   {\color{medgray}{.}}
  \end{textblock*} 
}

\newcommand{\twodots}[0]  
{  
  \begin{textblock*}{1cm}(1.136\textwidth,8.9cm)%
   {\color{medgray}{..}}
  \end{textblock*} 
}

\newcommand{\donedot}[0]  
{  
  \begin{textblock*}{1cm}(1.136\textwidth,8.9cm)%
   {\color{darkgray}{.}}
  \end{textblock*} 
}

\newcommand{\dtwodots}[0]  
{  
  \begin{textblock*}{1cm}(1.136\textwidth,8.9cm)%
   {\color{darkgray}{..}}
  \end{textblock*} 
}

\newcommand{\bdoton}[0]  
{  
  \begin{textblock*}{1cm}(1.136\textwidth,1.0cm)%
   {\color{medgray}{.}}
  \end{textblock*} 
}

% domains  --   --   --   --   --   --   --   --   --   --   --   --
\newcommand{\bigL}[0]    { L^{2} }
\newcommand{\littlel}[0] { l^{2} }

\newcommand{\innerxL}[2] { \langle #1 | #2 \rangle_{\bigL} }
\newcommand{\innerxl}[2] { \langle #1 | #2 \rangle_{\littlel} }

% action over domains
\newcommand{\domL}[0]    { \Omega }
\newcommand{\doml}[0]    { \sigma }

\newcommand{\intcont}[0] { \Omega = \lst{x\in\real{}\colon -1 \le x \le 1} }
\newcommand{\intdisc}[0] { \sigma = \lst{x\colon -1 \le x \le 1} }

\newcommand{\spaceL}[0]  { \bigL (\Omega) }
\newcommand{\spacel}[0]  { \littlel \paren{\sigma} }

\newcommand{\intL}[0]    { \bigL[-1,1]}
\newcommand{\intl}[0]    { \littlel[-1,1] }

\newcommand{\intdom}[1]  { \int_{\domL} #1(x) #1^{*}(x) dx }
\newcommand{\sumdom}[1]  { \sum_{x\in \doml} #1(x) #1^{*}(x) \Delta }

\newcommand{\intdomx}[2] { \int_{\domL} #1(x) #2^{*}(x) dx }
\newcommand{\intdomy}[2] { \bl{\int_{\domL}} #1(x) #2^{*}(x) dx }

\newcommand{\sumdomx}[2] { \sum_{x\in \doml} #1(x) #2^{*}(x) \Delta }
\newcommand{\sumdomy}[2] { \sum_{x\in \doml} #1(x) #2^{*}(x) \Delta }
\newcommand{\sumdomz}[2] { \bl{\sum_{x\in \doml}} #1(x) #2^{*}(x)\Delta }

%% Derivatives
\newcommand{\pd}[2]      { \frac{ \partial #1 } { \partial #2 } }
\newcommand{\pdt}[1]     { \pd{ #1 }{t} }
\newcommand{\pdx}[1]     { \pd{ #1 }{x} }

\newcommand{\td}[2]      { \frac{ d #1 } { d #2 } }
\newcommand{\tdt}[1]     { \td{ #1 } { dt } }
\newcommand{\tdx}[1]     { \td{ #1 } { dx } }

\newcommand{\pz}[0]      { \phantom{-1} }
\newcommand{\prx}[0]     { \pz 1 }
\newcommand{\pry}[0]     { \pz 0 }

\newcommand{\phitwo}[0]  {\phi(x_{1}, x_{2})}

%% Title page
\newcommand{\tpage}[0] 
{
\begin{frame}
  \begin{center}
    \bl{{\LARGE{Fixing Gauge and Rank Deficiency}}} \\[10pt]
    {\href{mailto:daniel.m.topa@erdc.dren.mil}{\large{Daniel Topa$^{1,2}$ and \href{http://math.unm.edu/people/academic-personnel/pedro-embid}{Pedro Embid}}$^{1}$}} \\[10pt]
    %
    \mg{{\scriptsize{$^{1}$ \href{http://www.math.unm.edu}{University of New Mexico, Department of Mathematics and Statistics}}}}\\[5pt]
    \mg{{\scriptsize{$^{2}$ \href{http://www.engilitycorp.com}{Engility Corporation}, 
    \\\href{http://www.usace.army.mil}{USACE} \href{http://www.erdc.usace.army.mil}{Engineer Research and Development Center},
    \\\href{http://www.erdc.usace.army.mil/Locations/InformationTechnologyLaboratory.aspx}{Information Technology Laboratory}}}}\\[10pt]
    %
    %{\normalsize{\href{http://www.siam.org/meetings/la15/}{SIAM Conference on Applied Linear Algebra}}} \\[5pt]
    \Put(-40,-100){\href{http://www.siam.org/meetings/la15/}{\includegraphics[ height = 2cm ]{../graphics/logos/"logo ala15"}}}
  \end{center}
\end{frame}
}

\endinput  %  -  -  -  -  -  -  -  -  -  -  -  -  -  -  -  -  -  -  -  -

%\tiny
%\scriptsize
%\footnotesize
%\small
%\normalsize
%\large
%\Large
%\LARGE
%\huge
%\Huge